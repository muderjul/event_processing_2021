For our evaluation we decided to mirror the papers evaluation as closely as possible
to see if we were able to replicate the results obtained there.
Therefore our baseline is the given implementation of RA in the paper.
Although we compare ourselves to the other implementation we decided to focus on
replicating the asymptotic results rather than beating the raw performance of their implementation.

\subsection{Experimental Setup}
As already mentioned we want to evaluate similarly to the paper, which means we also mostly varied
window sizes and used sliding granularity of 1 where not specified.
Further information on our data streams can be taken from our Data Generation section. % TODO LINK SECTION
To limit the tests and to gain the possibility of rerunning the tests with different configurations
and iterations of our implementation we limited ourselves to 5 million tuples and
a cap of 180 seconds spend only inside our reactive aggregator per run
which sums up to a maximum just short of 6 hours per run of the full evaluation suite.
We ran all tests on a 4-core 2.3 Ghz Intel Core i7 with 32 GB of RAM, using
Python 3.7.3 and MacOS 10.15.7

\subsection{Results and Analysis}
Since we were interested in replicating the papers results, we decided to
recreate figures (in the paper) 6, 8 and 10. Tables 3 and 4 as well as figure 11 were not suited
for such a comparison while our data sets were not able to guarantee the necessary
(1, 1)-changes for figure 7. Due to the previously mentioned possible bugs in our
implementation we decided to not replicate figure 9. We omitted comparing to figure 12
to keep our focus on the speed of the implementation instead of a memory footprint.
