
% VLDB template version of 2020-08-03 enhances the ACM template, version 1.7.0:
% https://www.acm.org/publications/proceedings-template
% The ACM Latex guide provides further information about the ACM template

\documentclass[sigconf, nonacm]{acmart}

\usepackage{float}
\usepackage{pgfplots}
\usepackage{subfiles}

%% The following content must be adapted for the final version
% paper-specific
\newcommand\vldbdoi{XX.XX/XXX.XX}
\newcommand\vldbpages{XXX-XXX}
% issue-specific
\newcommand\vldbvolume{14}
\newcommand\vldbissue{1}
\newcommand\vldbyear{2020}
% should be fine as it is
\newcommand\vldbauthors{\authors}
\newcommand\vldbtitle{\shorttitle}
% leave empty if no availability url should be set
\newcommand\vldbavailabilityurl{}
% whether page numbers should be shown or not, use 'plain' for review versions, 'empty' for camera ready
\newcommand\vldbpagestyle{plain}

\pgfplotsset{compat = newest}

\begin{document}
\title{ArgMax in Sliding-Window Aggregation}

%%
%% The "author" command and its associated commands are used to define the authors and their affiliations.
\author{Julian Muders}
\affiliation{%
  \institution{Humboldt-Universität zu Berlin}
  \city{Berlin}
  \country{Germany}
}
\email{mudersju@informatik.hu-berlin.de}

\author{Richard Herrmann}
\affiliation{%
  \institution{Humboldt-Universität zu Berlin}
  \city{Berlin}
  \country{Germany}
}
\email{herrmari@informatik.hu-berlin.de}

\author{Philipp Harnisch}
\affiliation{%
  \institution{Humboldt-Universität zu Berlin}
  \city{Berlin}
  \country{Germany}
}
\email{harnisph@informatik.hu-berlin.de}

%%
%% The abstract is a short summary of the work to be presented in the
%% article.
\begin{abstract}
\subfile{sections/abstract.tex}
\end{abstract}

\maketitle

%%% do not modify the following VLDB block %%
%%% VLDB block start %%%
\pagestyle{\vldbpagestyle}
\begingroup\small\noindent\raggedright\textbf{PVLDB Reference Format:}\\
\vldbauthors. \vldbtitle. PVLDB, \vldbvolume(\vldbissue): \vldbpages, \vldbyear.\\
\href{https://doi.org/\vldbdoi}{doi:\vldbdoi}
\endgroup
\begingroup
\renewcommand\thefootnote{}\footnote{\noindent
This work is licensed under the Creative Commons BY-NC-ND 4.0 International License. Visit \url{https://creativecommons.org/licenses/by-nc-nd/4.0/} to view a copy of this license. For any use beyond those covered by this license, obtain permission by emailing \href{mailto:info@vldb.org}{info@vldb.org}. Copyright is held by the owner/author(s). Publication rights licensed to the VLDB Endowment. \\
}\addtocounter{footnote}{-1}\endgroup
%%% VLDB block end %%%

%%% do not modify the following VLDB block %%
%%% VLDB block start %%%
\ifdefempty{\vldbavailabilityurl}{}{
\vspace{.3cm}
\begingroup\small\noindent\raggedright\textbf{PVLDB Artifact Availability:}\\

\endgroup
}
%%% VLDB block end %%%

\section{Goal and Structure}
\label{goal_and_structure}
\subfile{sections/goal_structure.tex}

\section{Applied Method}
\label{applied_methods}

\subfile{sections/implementation.tex}

\section{Obtained Results}
\label{obtained_results}

\subfile{sections/evaluation.tex}

\section{Discussion}
\label{discussion}

\subfile{sections/discussion.tex}

%%%%%%%%%%%%%%%%%%%%%%%%%%%%%%%%%%%%%%%%%%%%%%
% example of figure and ref from the template:
%\autoref{fig:duck}.
%\begin{figure}
%  \centering
%  \includegraphics[width=\linewidth]{figures/duck}
%  \caption{An illustration of a Mallard Duck. Picture from Mabel Osgood Wright, \textit{Birdcraft}, published 1897.}
%  \label{fig:duck}
%\end{figure}
%%%%%%%%%%%%%%%%%%%%%%%%%%%%%%%%%%%%%%%%%%%%%%

%%%%%%%%%%%%%%%%%%%%%%%%%%%%%%%%%%%%%%%%%%%%%%
% example of double column table from the template:
%\begin{table*}[t]
%  \caption{A double column table.}
%  \label{tab:commands}
%  \begin{tabular}{ccl}
%    \toprule
%    A Wide Command Column & A Random Number & Comments\\
%    \midrule
%    \verb|\tabular| & 100& The content of a table \\
%    \verb|\table|  & 300 & For floating tables within a single column\\
%    \verb|\table*| & 400 & For wider floating tables that span two columns\\
%    \bottomrule
%  \end{tabular}
%\end{table*}
%%%%%%%%%%%%%%%%%%%%%%%%%%%%%%%%%%%%%%%%%%%%%%

%%%%%%%%%%%%%%%%%%%%%%%%%%%%%%%%%%%%%%%%%%%%%%
% example of normal table from the template:
%\begin{table}[hb]% h asks to places the floating element [h]ere.
%  \caption{Frequency of Special Characters}
%  \label{tab:freq}
%  \begin{tabular}{ccl}
%    \toprule
%    Non-English or Math & Frequency & Comments\\
%    \midrule
%    \O & 1 in 1000& For Swedish names\\
%    $\pi$ & 1 in 5 & Common in math\\
%    \$ & 4 in 5 & Used in business\\
%    $\Psi^2_1$ & 1 in 40\,000 & Unexplained usage\\
%  \bottomrule
%\end{tabular}
%\end{table}
%%%%%%%%%%%%%%%%%%%%%%%%%%%%%%%%%%%%%%%%%%%%%%

%\begin{acks}
% This work was supported by the [...] Research Fund of [...] (Number [...]). Additional funding was provided by [...] and [...]. We also thank [...] for contributing [...].
%\end{acks}

%\clearpage

\bibliographystyle{ACM-Reference-Format}
\bibliography{sample}

\end{document}
\endinput
